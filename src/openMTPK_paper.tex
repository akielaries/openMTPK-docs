\documentclass[12pt, letterpaper]{article}

% must use this pkg for displaying imgs
\usepackage{graphicx}
\usepackage{ragged2e}
\usepackage{amsmath}
\usepackage{multirow,array}
\usepackage{blindtext}
\usepackage{etoolbox}
\usepackage{marginnote}
\usepackage[a4paper, total={6in, 9in}]{geometry}
\graphicspath{ {../../imgs/} }
% pkg for links
\usepackage{hyperref}

% for codeblocks highlighting
\usepackage{color}
\usepackage{xcolor}
\usepackage{listings}

% Solarized colour scheme for listings
\definecolor{solarized@base03}{HTML}{002B36}
\definecolor{solarized@base02}{HTML}{073642}
\definecolor{solarized@base01}{HTML}{586e75}
\definecolor{solarized@base00}{HTML}{657b83}
\definecolor{solarized@base0}{HTML}{839496}
\definecolor{solarized@base1}{HTML}{93a1a1}
\definecolor{solarized@base2}{HTML}{EEE8D5}
\definecolor{solarized@base3}{HTML}{FDF6E3}
\definecolor{solarized@yellow}{HTML}{B58900}
\definecolor{solarized@orange}{HTML}{CB4B16}
\definecolor{solarized@red}{HTML}{DC322F}
\definecolor{solarized@magenta}{HTML}{D33682}
\definecolor{solarized@violet}{HTML}{6C71C4}
\definecolor{solarized@blue}{HTML}{268BD2}
\definecolor{solarized@cyan}{HTML}{2AA198}
\definecolor{solarized@green}{HTML}{859900}
\definecolor{solarized@orchid}{HTML}{483D8B}


% Define C++ syntax highlighting colour scheme
\definecolor{commentgreen}{RGB}{2,112,10}
\definecolor{eminence}{RGB}{108,48,130}
\definecolor{weborange}{RGB}{255,165,0}
\definecolor{frenchplum}{RGB}{129,20,83}

\lstset {
    language=[GNU]C++,
    frame=tb,
    tabsize=4,
    showstringspaces=false,
    numbers=left,
    upquote=true,
    commentstyle=\color{commentgreen},
    keywordstyle=\color{solarized@orchid},
    identifierstyle=\color{solarized@blue},
    stringstyle=\color{solarized@magenta},
    basicstyle=\small\ttfamily, % basic font setting
    emph={int,char,double,float,unsigned,void,bool},
    emphstyle={\color{blue}},
    escapechar=\&,
    % keyword highlighting
    classoffset=1, % starting new class
    otherkeywords={TEST, EXPECT_EQ,>,<,.,;,-,!,=,~, typename},
    morekeywords={TEST, EXPECT_EQ,>,<,.,;,-,!,=,~, typename},
    keywordstyle=\color{frenchplum},
    classoffset=0
}


\makeatletter
\patchcmd{\raggedright}{\parindent\z@}{}{}{}
\makeatother

\begin{document}


\newcommand{\papertitle}{Akiel Aries}
%\newcommand{\paperminortitle}{a Linux-Focused Attack}
\newcommand{\papermajorheading}{openMTPK}
\newcommand{\paperminorheading}{open-source Mathematics Package}

\newcommand{\HRule}{\rule{\linewidth}{0.5mm}} % Defines a new command for the horizontal lines, change thickness here

\center % Center everything on the page

%----------------------------------------------------------------------------------------
%	HEADING SECTIONS
%----------------------------------------------------------------------------------------

\textsc{\Large \papermajorheading}\\[0.2cm] % Major heading such as course name
\textsc{\large \paperminorheading}\\[0.75cm] % Minor heading such as course title

%----------------------------------------------------------------------------------------
%	TITLE SECTION
%----------------------------------------------------------------------------------------

\HRule \\[0.4cm]
{ \papertitle}\\[0.1cm] % Title of your document
%{ \huge \paperminortitle}\\[0.025cm] % Title of your document


%----------------------------------------------------------------------------------------
%	AUTHOR SECTION
%--------------------------------------------------------------------------------------


%----------------------------------------------------------------------------------------
%	DATE SECTION
%----------------------------------------------------------------------------------------
{\large November 2022} \linebreak

\begin{sloppypar}

Abstract\\
\begin{flushleft}
\justify{
openMTPK is a reusable mathematics library written in C++ originally inspired from 
undergraduate coursework at Northern Arizona University in mathematics, cybersecurity 
and computer science, Python based projects in CS 499 Contemporary Developments, Deep
Learning, developments on vpaSTRM as well as my increasing curiosity of the all things 
pure and applied in the realm of mathematics\\
The goal is to make a reusable mathematics library similar to the use of math.h allowing
users to call within their own projects. Starting with some implementations of basics of 
different mathematical topics like arithmetic, calculus, statistics, linear algebra, etc. 
in conjunction with more advanced algorithms seen in the blend of such topics, branches 
of Machine Learning, image processing and much more. Some of the modules operate in a 
circular dependant way, for example arithmetic operations seen in linear algebra vector 
operations that can be seen in algorithms implemented into the deep learning module. Many of
these implementations were first prototypes in Wolfram Mathetmatica, then converted to C++ 
code for the package. Look in the drivers folder for examples on using these tools in your 
own project. This paper will highlight the modules in use, the mathematics/logic behind 
them, future developments, and more. \linebreak

github repository: \url{https://github.com/akielaries/openMTPK}

}
\newpage

\tableofcontents

\newpage

% classification img
%\includegraphics[scale=0.3]{mirai-major-events-timeline.png}
\section{Tools and Dependencies}
The goal of openMTPK is to have as little dependencies as possible without re-inventing the
wheel too much while performing speedy computations. Other than C and C++ standard libraries
the 3rd-party dependencies that are used are deemed necessary for many of the packages 
functionalities. 

\subsection{Tools}

\begin{itemize}
\item \textbf{gtest}: Used for unit testing. Within the projects development Makefile, 
there are option to run the tests for the modules in this package. This is done by compiling 
a driver file that runs the methods referenced in each module, the source module file itself, 
and then the gtest based test driver file for said module. Some modules are harder to test
than others but for the most part simply checking the expected output against the computed 
output does all the checking that is needed. 

\item \textbf{AFL and AFL++}: AFL is a fuzzing tool built by Google but now deprecated.
AFL++ is an open source community effort forked from AFL offering the same functionalities but 
much better. Advertised as better mutations, more speed, and better instrumentation. Fuzzing 
comes in handy for generating random inputs for your program and analyzing the outputs. 
Analyzing crashes, unique or not, proves valuable for implementing strong and secure code. 

The code below is an example on how testing on the arithmetic package was done.
\begin{lstlisting}
#include <openMTPK/arithmetic/arith.hpp>
#include <gtest/gtest.h>

namespace {
    arith ar;
    // test case, test name
    TEST(arith_test, add_positive) {
        EXPECT_EQ(46094, ar.add(93, 106, 3551, 42344));
        EXPECT_EQ(6.85, ar.add(1.25, 1.85, 2.75, 1));
    }
    // multiplication (product) testing
    TEST(arith_test, mult_positive) {
        EXPECT_EQ(240, ar.mult(10, 8, 3));
        EXPECT_EQ(6.359375, ar.mult(1.25, 1.85, 2.75, 1));
    }
    // subtraction
    TEST(arith_test, sub_positive) {
        EXPECT_EQ(5, ar.sub(10, 8, 3));
        EXPECT_EQ(1, ar.sub(1.25, 1.85, 2.75));
    }
}
\end{lstlisting}

\end{itemize}

\subsection{Dependencies}
openMTPK makes use of a few open source cross-platform compatible packages and libraries
in the cases of threading for performance, graphics libraries, and packages for testing
and fuzzing.
\begin{itemize}
\item \textbf{openMP}: Open Multi-Processing is useful for simple threading when needed.
For loops that don't make complex calls can be encosed in a 
\verb|#pragma omp parallel for {}| declaration.

\item \textbf{openCL}: Open Computing Language is another useful threading API that allows for
more customized parallelization techniques. 

\item \textbf{openGL}: Open Graphics Library used for hardware-accelerated rendering. API
makes use of interaction with a machines Graphics Processing Unit (GPU) allowing for quick
rendering of modules that make use.

\item \textbf{libXbgi}: Borland Graphics Interface reiteration. Provides a useful graphics API
to get started with visualizing mathematical algorithms.

\end{itemize}


\section{Modules}
The purpose of this package is to also be as modular as possible. Simply calling a module in 
the package from within your C++ project with \verb|#include <openMTPK/arithmetic/arith.hpp>|, 
this allows a developer to utilize the basic operations implemented in the arithmetic module.
Calling \verb|#include <openMTPK/number_theory/primes.hpp>| allows a developer to use the 
prime number based operations within the number theory module. \linebreak
As of now (12/2022), there is no formal installation process or real way to use openMTPK.
The use of Makefiles + shell scripts is primarily for development purposes, users of the 
package will need to compile their own version of the library in their standard
library directories or include the source and header files in their projects with some 
modifications. \linebreak
openMTPK is available in a range of programming language bindings. As of Dec. 2022, the 
Python API is developed the furthest with R being the next. The APIs for Fortran and OCaml 
Perhaps in later implementations of openMTPK, APIs for Python, Julia and possibly MATLAB 
depending on how easy the binding process is. 

\subsection{Arithmetic}
Basic arithmetic operations, addition, subtraction, multiplication, exponentiation, were 
implemented in a way to accept `n` arguments of `n` types. Meaning calling any of the 
respective functions allows numerous parameters to be passed in of various numeric data type. 
This acts as the basis for many of the basic functionalities of this package.
In the repository for the project, see \verb|/include/arithmetic/arith.hpp| for the 
implementation of this module. Since it makes use of C++ class templates, this module is 
header-only in development. Addition, subtraction, and multiplication were all implemented 
in an identical way, making use of templates and inline functionalites. Within the openMTPK 
repository, I had provided examples on how to use the modules as seen in the 
\verb|/drivers| folder. Each driver file contains a \verb|main()| function.

\subsubsection{Addition}
\noindent The arithmetic operation addition can sometimes be thought of the adding of two 
digits while a summation is the adding of a series of two or more digits. openMTPK's add() 
function makes use of both. Since it is a template class allowing to add n numbers of many 
numeric data types, passing in just two numbers to add() qualifies as plain addition while 
passing in two or more variables qualifies as more simplified version of a summation.\\

\noindent Below is how the addition operation was implemented.
\begin{lstlisting}
template<typename T>
inline T add(T t) {
	return t;
}
template<typename T, typename... Ts>
inline auto add(T t, Ts... ts) {
	return t + add(ts...);
}
\end{lstlisting}

\noindent\textbf{Addition}:
\begin{flalign*}
&z = x + y&
\end{flalign*}
Where \textbf{x} and \textbf{y} are the addends and \textbf{z} is the sum \\

\noindent A typical summation may look something like this: \\
\textbf{Summation}: 
\begin{flalign*}
f(x) & =\sum_{i=1}^{n} x_1 + x_2 + ... + x_n&
\end{flalign*}

\noindent Where \textbf{i} is the index of the summation and \textbf{n} is the upper 
limit. For example the summation of one with a limit of three would look like:
\begin{flalign*}
&6 = 1 + 2 + 3&
\end{flalign*}

\noindent openMTPK's add() function operates differently for multiple digits as it does not
take in an upper limit to calculate up to from the index. Instead, it takes in multiple
indices like a list and adds them together. See the formula below:\\
\begin{flalign*}
f(x) & =\sum_{i=a} a_1 + a_2 + ... + a_n&
\end{flalign*}

\noindent Where \textbf{a} is a series of digits input into the call to the function and 
\textbf{n} is the index of the list-like series passed in. See below:
\begin{lstlisting}
int sum = ar.add(4, 8, 1, 2);
\end{lstlisting}

\noindent Translating the function call into the above formula would look like:
\begin{flalign*}
& 15 = 4 + 8 + 1 + 2&
\end{flalign*}

Below is a brief example on how to use the multiplication function in the arithmetic 
module. 
\begin{lstlisting}
#include <iostream>
#include <stdio.h>
#include <vector>
#include <cassert>
#include <openMTPK/arithmetic/arith.hpp>

int main() {
    // declare arithmetic class object
    arith ar;
    int a = 10;
    int b = 8;
    int c = 3;
    double d = 1.25;
    float e = 1.85;
    float f = 2.75;
    long g = 1.35;
    float r0 = ar.add(a, b, c);
    int r1 = ar.add(d, e, f, g);
    float r2 = ar.add(d, e, f, g);

    printf("%d + %d + %d = %f\n", 
			a, b, c, r0);
    printf("%f + %f + %f + %ld = %d\n", 
    			d, e, f, g, r1);
    printf("%f + %f + %f + %ld = %f\n\n", 
    			d, e, f, g, r2);
    return 0;	
}
\end{lstlisting}

\subsubsection{Subtraction}
This operation works exactly like \verb|add()| except it is performing subtraction 
instead of addition. 
\begin{flalign*}
&z = x - y&
\end{flalign*}
Where \textbf{x} is the \textit{minuend} \textbf{y} is the \textit{subtrahend} making
\textbf{z} the \textit{difference}. \\

\noindent Below is a brief example on how to use the subtraction function in the arithmetic 
module. 
\begin{lstlisting}
#include <iostream>
#include <stdio.h>
#include <vector>
#include <cassert>
#include <openMTPK/arithmetic/arith.hpp>

int main() {
    // declare arithmetic class object
    arith ar;
    int a = 10;
    int b = 8;
    int c = 3;
    double d = 1.25;
    float e = 1.85;
    float f = 2.75;
    long g = 1.35;
    int r3 = ar.sub(a, b, c);
    int r4 = ar.sub(d, e, f ,g);
    float r5 = ar.sub(d ,e, f, g);
    
    printf("%d - %d - %d = %d\n", a, b, c, r3);
    printf("%f - %f - %f - %d = %d\n", d, e, f, g, r4);
    printf("%f - %f - %f - %f = %f\n\n", d, e, f, g, r5);
    return 0;	
}
\end{lstlisting}


\subsubsection{Multiplication}
\noindent The arithmetic operation multiplication can be conceptualized the same as 
addition. A product \begin{math}\prod \end{math} is the multiplication of a series 
of two or more numbers. Just like summation there are typically an index and an upper 
limit. Identical to the addition operation of this module, multiplication behaves like
a modified product formula. 

\noindent Below is how the multiplication operation was implemented.
\begin{lstlisting}
template<typename W>
inline W mult(W w) {
	return w;
}
template<typename W, typename... Wv>
inline auto mult(W w, Wv... wv) {
	return w * mult(wv...);
}
\end{lstlisting}

\newpage

\noindent\textbf{Multiplication}:
\begin{flalign*}
&z = x \cdot y&
\end{flalign*}
Where \textbf{x} is the \textit{multiplier} \textbf{y} is the \textit{multiplicand} 
making \textbf{z} the \textit{product}.

\noindent A typical summation may look something like this: \\
\textbf{Product}: 
\begin{flalign*}
f(x) & =\prod_{i=1}^{n} x_1 \cdot x_2 \cdot ... \cdot x_n&
\end{flalign*}

\noindent Where \textbf{i} is the index of the product and \textbf{n} is the upper 
limit. 

\noindent For example the product of one with a limit of three would look like:
\begin{flalign*}
&6 = 1 \cdot 2 \cdot 3 &
\end{flalign*}

\noindent Take this sample into account as well.
\begin{flalign*}
f(x) & =\prod_{i=1}^{6} i^2&
\end{flalign*}

\noindent This results in:
\begin{flalign*}
&518,400 = 1 \cdot 4 \cdot 9 \cdot 16 \cdot 25 \cdot 36&
\end{flalign*}

\noindent openMTPK's mult() function operates differently for multiple digits as it does not
take in an upper limit to calculate up to from the index. Instead, it takes in multiple
indices like a list and multiplies them together. See the formula below:\\
\begin{flalign*}
f(x) & =\prod_{i=a} a_1 \cdot a_2 \cdot ... \cdot a_n&
\end{flalign*}

\noindent Where \textbf{a} is a series of digits input into the call to the function and 
\textbf{n} is the index of the list-like series passed in. See below:
\begin{lstlisting}
int sum = ar.mult(4, 8, 1, 2);
\end{lstlisting}

\noindent Translating the function call into the above formula would look like:
\begin{flalign*}
& 64 = 4 \cdot 8 \cdot 1 \cdot 2&
\end{flalign*}

\noindent Below is a brief example on how to use the multiplication function in the 
arithmetic module. 
\begin{lstlisting}
#include <iostream>
#include <stdio.h>
#include <vector>
#include <cassert>
#include <openMTPK/arithmetic/arith.hpp>

int main() {
    // declare arithmetic class object
    arith ar;
    int a = 10;
    int b = 8;
    int c = 3;
    double d = 1.25;
    float e = 1.85;
    float f = 2.75;
    long g = 1.35;
    int r6 = ar.mult(a, b, c);
    int r7 = ar.mult(d, e, f ,g);
    double r8 = ar.mult(d ,e, f, g);

    printf("%d * %d * %d = %d\n", a, b, c, r6);
    printf("%f * %f * %f * %d = %d\n", d, e, f, g, r7);
    printf("%f * %f * %f * %ld = %f\n\n", d, e, f, g, r8);
    return 0;	
}
\end{lstlisting}


\subsubsection{Exponentiation}
\noindent Exponentiation is also a basic arithmetic operation that also makes use of
the multiplication operation. Multplying the base by itself continuously as specified
by the power. STL provides the \verb|pow()| function that is capable of performing similar
operations, openMTPK's \verb|exp()| function operates in similar ways to the previously
mentioned \verb|add()|, \verb|sub()| and \verb|mult()| functions, taking in \textbf{n} 
arguments of \textbf{n} types. A basic exponentiation would include one base and one 
power, openMTPK's exponentiation function allows for one base and \textbf{n} powers. \\

\noindent\textbf{Exponentiation}:
\begin{flalign*}
& x^n = \underbrace{x \cdot x \cdot x \cdot ... \cdot x}_\textrm{n times} &
\end{flalign*}

\noindent Where \textbf{x} is the raised to the power of \textbf{n}. \\

\noindent Say we have an exponential equation like below: 
\begin{flalign*}
& x^{n^i} &
\end{flalign*}
Where \textbf{x} is our base, \textbf{n} is the power of our base, and \textbf{i} is the
power of the result of our original exponential equation. Breaking it down as such:
\begin{flalign*}
& x^{n^i} \Rightarrow \underbrace{x \cdot x \cdot ... \cdot x}_\textrm{n times} = y^i & \\
& y^i \Rightarrow \underbrace{y \cdot y \cdot ... \cdot y}_\textrm{i times} = z &
\end{flalign*}
\noindent Where \textbf{z} is our result.

\noindent Below is how the exponentiation operation was implemented.
\begin{lstlisting}
template<typename Z>
inline W exp(Z z) {
	return z;
}
template<typename Z, typename... Zy>
inline auto exp(Z z, Zy... zy) {
	return z *= exp(zy...);
}
\end{lstlisting}


\subsection{Calculus}
Calculus is a branch of mathematics focused on continuous change. Split into two branches
(see below), concerned with rates of change, slopes of a given curve, quanitity accumulations,
areas between and below curves.

\subsubsection{Differential} 
The derivative of a function of a real variable measures the sensitivity to change of the 
function value (output value) with respect to a change in its argument (input value). For 
example, the derivative of the position of a moving object with respect to time is the 
object's velocity: this measures how quickly the position of the object changes when time 
advances.
\subsubsection{Integral}
Coming Soon...

\subsection{Linear Algebra}
Branch of mathematics concerned with linear equations, maps and their purposes and value 
in vector and matrix spaces. The concept is foundational in other areas of mathematics, 
computing, and other realms of science and engineering. 

\subsubsection{Vector Addition}
Two vectors of the same size can be added together by adding the corresponding elements, to 
form another vector of the same size, called the sum of the vectors. Vector addition is 
denoted by the symbol +. (Thus the symbol + is overloaded to mean scalar addition when 
scalars appear on its left- and right-hand sides, and vector addition when vectors appear 
on its left- and right-hand sides.)

\subsubsection{Vector Subtraction}
Coming Soon...

\subsubsection{Vector Multiplication}
Another operation is scalar multiplication or scalar-vector multiplication, in which a vector 
is multiplied by a scalar (i.e., number), which is done by multiplying every element of the 
vector by the scalar. Scalar multiplication is denoted by juxtaposition, typically with the 
scalar on the left.

\subsection{Number Theory}
A branch of pure mathematics with applications in cryptography, phyics, others areas of 
computation, etc. that primarily deals with the study of integers, integer valued functions, 
prime numbers, as well as the attributes and properties made from integers and their functions. 
This topic of mathematics is quite theoretical and is difficult to view the applications in a 
comprehensible sense. Beyond the basics of Number Theory, this module will also focus on 
encryption algorithms.

\subsubsection{Caesar Cipher}
Coming Soon...

\subsubsection{Monoalphabetic Substitution Keyword Cipher}
\subsubsection{Rivest Cipher 2 (RC2) Encryption Algorithm}
\subsubsection{Rivest Cipher 4 (RC4) Encryption Algorithm}
A stream cipher algorithm created by Ron Rivest (creator of RSA) previously declared 
insecure. RC4 generates a pseudorandom stream of bits (a keystream). As with any stream 
cipher, these can be used for encryption by combining it with the plaintext using bitwise 
exclusive or; decryption is performed the same way (since exclusive or with given data is 
an involution). This is similar to the one-time pad, except that generated pseudorandom 
bits, rather than a prepared stream, are used. To generate the keystream, the cipher makes 
use of a secret internal state which consists of two parts:

\item 1. A permutation of all 256 possible bytes (denoted "S" below).
\item 2. Two 8-bit index-pointers (denoted "i" and "j").
The permutation is initialized with a variable-length key, typically between 40 and 2048 
bits, using the key-scheduling algorithm (KSA). Once this has been completed, the stream 
of bits is generated using the pseudo-random generation algorithm (PRGA).

\subsubsection{Rivest Cipher 5 (RC5) Encryption Algorithm}
Coming Soon...

\subsubsection{Rivest Cipher 6 (RC6) Encryption Algorithm}
Coming Soon...

\subsection{Deep Learning}
\subsubsection{Linear Regression}
A statistical algorithm with a linear approach for modeling the relationship between a 
scalar response and one or more explanatory variables (also known as dependent and 
independent variables). In linear regression, the relationships are modeled using linear 
predictor functions whose unknown model parameters are estimated from the data. Such models 
are called linear models.

\subsubsection{Cross Validation}
A resampling technique, the idea of this method is to train our model by utilizing 
the subsets of our data set then proceeding to evaluate + compare against the original.
Essentially, use some part(s) of the data set for training, other part(s) for testing.

\subsubsection{K-Fold Cross Validation}
Split our data into a `k` number of subsets also called folds, and perform
training/learning on the subsets leaving one `(k - 1)` for the final evaluation of 
the trained model. The method involves iterating `k` number of times using a different
fold each time for testing against.

\subsection{Complex}
I had named this module accordingly due to its overall complexity and difficulty. This module
name will be more clear as some topics will include the name. This module makes use of the 
openGL(Open Graphics Library) package for graphics purposes. 

\subsubsection{Topology}
Topology is concerned with the properties of a geometric object that are preserved under
continuous deformations, such as stretching, twisting, crumpling, and bending; that is,
without closing holes, opening holes, tearing, gluing, or passing through itself.

\subparagraph*{Torus}: A torus is a geometrical disc-shaped figure that is created by
revolving a circle in a three-dimensional space about an axis that is coplanar with the 
cicle. Depending on the revolutions of the circle, different types of tori can be formed. 
Horn, spindle, and more types of torus are made when revolutions into a spherical disc 
are not met.

\subsubsection{Dynamical Systems}
Systems in which a function describes the time dependence of a point in an ambient space. 
\subparagraph*{Fractals} A geometric shape containing detailed structure at arbitrarily 
small scales, usually having a fractal dimension strictly exceeding the topological 
dimension. \\

\begin{itemize}
\item \textbf{Mandelbrot Fractal}: \\

\item \textbf{Julia Fractal}: \\

\end{itemize}


\subsubsection{Spline}
A spline curve is a mathematical representation for which it is easy to build an interface 
that will allow a user to design and control the shape of complex curves and surfaces.
The Complex module includes ways to create your own Non-Uniform Rational B-Splines (NURBS),
Bezier Surfaces (as well as patches and curves)


\end{flushleft}
\end{sloppypar}
\end{document}
